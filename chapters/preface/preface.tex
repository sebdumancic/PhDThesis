\chapter*{Acknowledgements}                                  \label{ch:preface}



Looking back at the past four years, I realise choosing to enrol in a PhD was the right choice.
Pursuing the PhD has been a wish of mine for a long time, and this experience came with many memories and very few regrets.
That is largely due to people that were involved with it, and I want to thank them all.








First and foremost, I would like to thank Hendrik for giving me the opportunity to do research.
His advice has helped me to become a better researcher, even though I might have not realised it at the time.
He taught me the importance of always keeping an eye on the big picture while still getting every detail right, no matter how small it is.
More importantly, he taught me the importance of clear and effective communication.
This was a rather hard lesson for me, and Hendrik knows that best: as a starting PhD student I was notoriously uninterested in providing concrete examples for any abstract idea I have had (at the end, no example can cover every aspect of a complex idea, no?).
I am sure it took a significant effort just to bare with that.
His (sometimes necessarily harsh) feedback during the first years has fortunately taught me better.




I am also grateful to the members of my jury.
Discussion with Jesse Davis and Johan Suykens over the past four years have pointed me towards a research idea I was unaware of; these discussions have helped me shape the work in this thesis.
I am thankful to Mathias Niepert, David Poole and Sebastian Riedel, not only for agreeing to join my jury: they might not be aware it, but they are the first three non-KU Leuven people that ever showed interest in my research.
When I met them in New York, at the StarAI workshop in 2016, I have been struggling to publish my first paper; their genuine interest provided encouragement needed to continue pursuing my ideas.
Therefore, it is more than fitting for them to be a part of my jury.
Finally, I would like to thank Prof. Carlo Vandecasteele for chairing the public defence, and Prof. Jan Van Humbeeck for chairing the private defence.




I have been lucky to be involved in several collaborations during my PhD.
Each of them taught me a lot.
I have enjoyed working on COBRAs with Toon, and was envious of his ability to produce a visualisation of any problem we were facing.
It also reminded me to always expect the unexpected -- this idea came unexpectedly from simply sitting in the same office, and many solutions we found came from unexpected places.
My only regret is that we never managed to move to other snakes.
Wannes was always the \textit{guy with solutions}; I am thankful for all of the tips \& tricks I've learned from him.
I remember well insightful discussions with Tias, not only about the underpinnings of CP, but many unexpected topics that popped up in the meantime.
His enthusiasm about, well, everything keeps any discussion vivid.
The DeepProbLog journey with Robin, Thomas, Angelika and Luc has taught me a lot about connecting dots that are seemingly difficult to relate and the benefit of having various perspectives on the table.
I have enjoyed my time in Heidelberg with Mathias, Alberto, Anja and Daniel.
This visit has helped me broaden the perspective on the ideas explored in this thesis, and has also opened new avenues to pursue.
Finally, Pedro has shown me how a challenge can make things fun.






I was lucky to pursue my PhD in a group where personal interactions matter.
It was a great pleasure to meet all of them, and that on its own has made the experience enjoyable.
I am thankful to Jessa for being a true friend, a person to talk to, and for binding the group together.
Equally importantly, for introducing me to the secrets of Decadenza and sharing the enthusiasm (as you would expect from a food snob).
To Vova, for (astonishingly) various conversations and all the follow-up e-mails about equally diverse information.
Especially for the intricacies of a hedgehog behaviour.
To Irma, for her positive energy and for being the initiator of many things we enjoyed.
To Toon, Elia, Vincent N. and Antoine, for the excellent atmosphere in the office which has stimulated many discussions, work- and fun-related ones equally.
I still think this is the best office in the building.
To Behrouz, for always showing a different perspective (and for Faloodeh).
To Angelika and Guy, for always willing to discuss my ideas.
To Pedro, for challenging every opinion (and for squash).
To Ondrej and Benjamin, for their grumpiness that gave a special flavour to every discussion.
To Sergey, for being so random and to Irina for putting up with him.
To Anna, Anton, Antoine, Arcchit, Arne, Bogdan, Evgeniya, Francesco, Guillermo, Jonas, Joris, Kurt, Killian, Kshitij, Leander, Leen, Mohit, Nitesh, Robin, Ruben, Samuel, Siegfried, Stella, Tom, Vaishak, Vincent (the big one), Thanh, Wen-Chi and others for many memorable coffee breaks, Secret Santas, movies and the Escape Room in Ypres.




Some people left the mark in an unexpected (even unintended) way.
I am thankful to Nada Lavrač for introducing me to the field SRL during my visit to her lab in Ljubljana; this has convinced me that logic is not as boring as I used to think.
To Jan Šnajder, for being an enthusiastic AI professor that initiated my interest in AI.
To Oliver Schulte, for being the first person that ever approached me at a conference (and doing so willingly).
To Alex, Ana and Marielle, for the Summer School in Cadiz.
To Carolina and Daniel, for my Erasmus year in Leuven.




Summer School of Science has played an important role throughout the years of my PhD.
This is probably the most influential experience that persuaded me to start a career in science.
I am thankful to people that made it such a special place: Anamarija, Anna-Maria, Jelena, Dora, Dunja, Kristina, Korado, Leonardo,  Mario, Marko, Matija, Matilda, Neva, Nikolina, Petras, Renan, and many others.





One regret I have is that I was never particularly good at maintaining old relationships, especially the long-lasting ones from Croatia.
But going back to Croatia for Summer and Christmas, I was always looking forward to meeting Viktor, Ivan, Matko and Domagoj.
We might not see each other for a while, but I sincerely appreciate that it always feels like we saw each other recently.




A special gratitude goes to my parents, Ljubica and Dražen, for their support over the years.
It was not always easy to put up with the idea my brothers and I could come up with, but they have always encouraged us to achieve everything we envisioned.
They taught us how to be persistent and not give up when it becomes difficult.



Finally, to Nina.
I don't have enough words to thank you.
Thank you for entering my life unexpectedly and changing it for the better.
For making me the person I am today.
For all meows, songs and goofy moments; sushi, curries and Ben 'n' Jerry's; travels, hidings around the apartment and evenings on the couch; Bjorn, Bocko and Robert; for many things I'm forgetting right now.
And for making our little oasis in Schipvaartstraat a home.



\begin{flushright}
  Sebastijan Dumančić \\
  Leuven, Belgium \\
  November 2018
\end{flushright}















%%%%%%%%%%%%%%%%%%%%%%%%%%%%%%%%%%%%%%%%%%%%%%%%%%
% Keep the following \cleardoublepage at the end of this file,
% otherwise \includeonly includes empty pages.
\cleardoublepage

% vim: tw=70 nocindent expandtab foldmethod=marker foldmarker={{{}{,}{}}}
