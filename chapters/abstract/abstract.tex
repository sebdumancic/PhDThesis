\chapter{Abstract}                                 \label{ch:abstract}

The early 21$^{st}$ century has been largely shaped by the huge amounts of available data generated by the widespread adoption of information technology.
This rapid growth of the stored data has created the need for automated tools capable of extracting useful bits from large amounts of data.
In turn, such tools have changed our perspective on data from a mere record of an event to a carrier of useful information.
Before insights can be drawn from the data, the data has to be first brought into a suitable form by means of \textit{feature engineering} as it can rarely be processed in its \textit{raw} form.
This step is usually time- and labour-intensive, and often requires an extensive knowledge of the domain.


In this thesis, we tackle the problem of \textit{representation learning} -- how to automate the process of feature construction?
We focus on feature construction with rich relational data expressed in form of networks, e.g., biological and traffic networks, as many real-life problems can be easily expressed in this format.
To be able to express complex feature over networks, the proposed methods rely on the expressive data representation language of first-order logic.


The first contribution of the thesis is a new versatile relational clustering framework that decouples various sources of relational similarity and combines them in a systematic manner.
The second contribution is CUR$^2$LED -- a relational representation learning framework that exploits approximate symmetries in relational data as features.
The third contribution are \textit{auto-encoding logic programs}  -- a relational generalisation of auto-encoders, one of the basic representation learning primitives.
The fourth contribution is the experimental comparison of various relational representation learning methods that offers insights into the strengths and weaknesses of the existing approaches.


Experiments demonstrate the promise of relational representation learning methods as a building block in any relational machine learning framework as the relational classifiers enhanced with the method introduced in the thesis improve the performance of relational classifiers.



%%%%%%%%%%%%%%%%%%%%%%%%%%%%%%%%%%%%%%%%%%%%%%%%%%
% Keep the following \cleardoublepage at the end of this file,
% otherwise \includeonly includes empty pages.
\cleardoublepage

% vim: tw=70 nocindent expandtab foldmethod=marker foldmarker={{{}{,}{}}}
