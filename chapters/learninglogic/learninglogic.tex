\chapter{Learning with logic}\label{ch:learninglogic}



% TODO: the previous chapter should clearly motivate the importance of representation, focus of the thesis

% TODO: describe a few examples of data (tabular, excel, networks, programs)

% TODO: what is a general language to represent all of it?





\section{Representing data with logic}


% TODO: syntax (clausal logic?)

% TODO: semantics

% TODO: cover the examples in logical format (focus on the strengths of logic)



\subsection{Hierarchy of representation languages}

% TODO: simplify chapter from Luc's book

% TODO: example (poker from Guy?)



\section{Inductive logic programming}


% TODO: machine learning typically done on  tabular data, that does not work for us

% TODO: define ILP problem 


\subsection{Learning as Search}

% TODO: illustrate the main loop with example


\subsubsection{Case study: TILDE}

\subsubsection{Case study: Aleph}


\section{Probabilistic logic programming}

% TODO: motivation: logic is too strict, we can benefit from uncertainty













%%%%%%%%%%%%%%%%%%%%%%%%%%%%%%%%%%%%%%%%%%%%%%%%%%
% Keep the following \cleardoublepage at the end of this file, 
% otherwise \includeonly includes empty pages.
\cleardoublepage

% vim: tw=70 nocindent expandtab foldmethod=marker foldmarker={{{}{,}{}}}
