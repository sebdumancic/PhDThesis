\chapter{Beknopte samenvatting}

Het begin van de 21ste eeuw wordt gekenmerkt door de grote hoeveelheden beschikbare data die geproduceerd worden door de algemene ingebruikname van informatietechnologieën. Deze snelle groei aan opgeslagen data zorgt voor een nood aan geautomatiseerde toepassingen om nuttige informatie uit de grote hoeveelheden  data te onttrekken. Zulke toepassingen hebben op hun beurt ons perspectief op data veranderd van  een medium om gebeurtenissen op te slaan tot een drager van nuttige informatie. Voordat inzichten uit data verkregen kunnen worden, moeten de data eerst in een geschikte vorm gebracht worden door middel van feature engineering, aangezien deze slechts zelden in hun ruwe vorm kunnen worden verwerkt. Deze stap is meestal tijds- en arbeidsintensief en vereist vaak een uitgebreide domeinkennis.

In dit proefschrift behandelen we het probleem van het leren van representaties – hoe kan het feature-constructieproces geautomatiseerd worden? We focussen op feature-constructie met rijke relationele data uitgedrukt in de vorm van netwerken, aangezien veel alledaagse problemen gemakkelijk kunnen uitgedrukt worden in dit formaat, bijvoorbeeld biologische netwerken en verkeersnetwerken. Om complexe features van netwerken te kunnen uitdrukken, maken de voorgestelde methodes gebruik van eerst-orde logica als expressieve taal om de data voor te stellen.
 
De eerste bijdrage van dit proefschrift is een nieuwe veelzijdig framework voor relationele clusteranalyse. Dit framework ontkoppelt verschillende bronnen van similariteit en combineert ze op een systematische manier. De tweede bijdrage is CUR$^2$LED – een framework voor het leren van relationele representaties dat benaderde symmetrieën in de relationele data gebruikt als features. De derde bijdrage zijn auto-encoderende logische programma’s – een relationele veralgemening van auto-encoders, een van de basisprimitieven voor relationeel leren. De vierde bijdrage is een experimentele vergelijking tussen verschillende methodes om relationele representaties te leren. Deze vergelijking biedt inzichten in de sterktes en zwaktes van de bestaande manieren.

Experimenten tonen het potentieel aan van methodes voor het leren van relationele representaties als bouwsteen in eender welk framework voor relationeel machinaal leren. Dit potentieel wordt aangetoond doordat de relationele classificatiesystemen uitgebreid met de methodes geïntroduceerd in dit proefschrift de performantie van relationele classificatiesystemen verbeteren.


%%%%%%%%%%%%%%%%%%%%%%%%%%%%%%%%%%%%%%%%%%%%%%%%%%
% Keep the following \cleardoublepage at the end of this file, 
% otherwise \includeonly includes empty pages.
\cleardoublepage

% vim: tw=70 nocindent expandtab foldmethod=marker foldmarker={{{}{,}{}}}
